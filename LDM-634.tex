\documentclass[DM,lsstdraft,toc]{lsstdoc}


% lsstdoc documentation: https://lsst-texmf.lsst.io/lsstdoc.html

% Package imports go here.
\usepackage{enumitem}

% Local commands go here.

\newcommand{\usecase}[2]{\subsection{\emph{#1}: #2}\label{use:#1}}

\newenvironment{flow}[1][Basic flow]
  {\subsubsection*{#1}\begin{enumerate}[label=\alph*.,itemsep=0pt]}
  {\end{enumerate}}

% To add a short-form title:
% \title[Short title]{Title}
\title{Production Campaign Management Services Use Cases}

% Optional subtitle
% \setDocSubtitle{A subtitle}

\author{Mikolaj Kowalik, Michelle Gower, and Rob Kooper}

\setDocRef{LDM-634}

\date{\today}

% Optional: name of the document's curator
% \setDocCurator{The Curator of this Document}

\setDocAbstract{%
Abstract text.
}

% Change history defined here.
% Order: oldest first.
% Fields: VERSION, DATE, DESCRIPTION, OWNER NAME.
% See LPM-51 for version number policy.
\setDocChangeRecord{%
  \addtohist{1}{YYY-MM-DD}{Unreleased.}{Mikolaj Kowalik}
}

\begin{document}

% Create the title page.
% Table of contents is added automatically with the "toc" class option.
\maketitle

\section{Introduction}

This document describes use cases for campaign management service.

\section{Actors}

\begin{description}
  \item[Operator]
    Batch production operator for DRP (has additional privileges).
\end{description}

\section{Campaign management}

\usecase{CM1}{Initiate campaign}
Campaigns are initiated in response to an LSST objective, by specifying an
initial set of pipelines, a coverage set, and an initial priority.  The Batch
Production Service is consulted with a reasonable lead time.  Consistent with
LSST processes, pipelines can be modified or added (for example, in the case of
after-burners) during a campaign. These changes and additions are admitted when
the criteria of change control processes are satisfied, including
\begin{itemize}
  \item
    relevant build-and test criteria
  \item
    the impact of resource-intensive campaigns is approved and understood
  \item
   production-scale test campaigns
\end{itemize}

\usecase{CM2}{Terminate failed campaign}
Reasons for a campaign failure will be documented and submitted to Science
Operations for review. Deletion of data products needs to be scheduled so that
it occurs after the review is completed. This includes backing out files,
materials from databases, and other production artifacts from the Data
Backbone, and maintaining production records as these activities occur.

\usecase{CM3}{Pause campaign}
Stop a long running campaign from proceeding allowing for TBD interventions.

\usecase{CM4}{Deal with problematic campaign}
LSST is a large system. Pipelines will evolve and be maintained. There will be
the campaigns, described in the operations documents. It is the nature of the
system that as issues emerge extra resources will be needed to provide focused
scrutiny on aspects of production for some pipeline. In many cases problems
will be resolved by bug fixes, or addressed by quality controls and changes to
processes. Any system needs to support mustering focused effort on quality
analysis that is urgent, and lacks an adequate basis for robust quality
controls.  The LSST Data Facility Batch Production Services staff contribute
effort to to solve these problems, in collaboration with Science Operations (or
other parties responsible for codes).

\usecase{CM5}{Deal with defective data}
Production data may be deemed defective immediately as the associated pipelines
terminate or after a period of time when inspection processes run.  Such data
need to be marked such that they will not be included in release data and will
be set aside for further analysis.

\usecase{CM6}{Deal with sudden lack (or surplus) in resources}
As noted above, for large scale computing, the amount of resource available to
support all campaigns will vary due to scheduled and unscheduled outages.

The technical system responds to an increase or decrease in resources by
running more or few jobs, once the workload manager is aware of the new level
of resources. The technical system responds to hardware failures on a running
job in just like any other system -- with the ultimate recovery being to delete
an partial data and retry, while respecting the priorities of the respective
campaigns.

\usecase{CM7}{Respect campaign priority}
Each campaign has an adjustable campaign priority reflecting LSST priority for
that objective. The service needs to support reliable operation of an ensemble
of many campaigns, respecting those priorities.

\usecase{CM1}{Monitor computing resources}
The Operator checks which computing platforms are available for a campaign
execution.
\begin{flow}
  \item
    The Campaign Manager periodically gathers data regarding an availability of
    a set of computing resources.
  \item
    It detects unscheduled downtimes of the resources.
  \item 
    It reports back to the operator current status of the resources.
\end{flow}

\usecase{CM1}{Monitor executions of campaigns}
The Operator checks the status of dispatched campaigns.
\begin{flow}
  \item
    The Campaign Manager periodically gathers real-time data regarding
    \begin{itemize}
      \item
        number of running, pending, completed, and failed jobs for each campaign
      \item
        resource usage for individual, running jobs (memory, data volume)
      \item
        failed pipelines and campaigns
    \end{itemize}
  \item
    After data are gathered, the Campaign Manager exposes these data to the
    Operator.
\end{flow}

\usecase{CM1}{Restart a failed pipeline}
The Operator after resolving issues dispatches a failed pipeline again to the
WMS to start it from the last point of failure \emph{without} waiting for a
campaign to complete a first run through of all pipelines.

\usecase{CM1}{Allow for parameter overrides}
The Operator decides to run a campaign overriding a subset of configuration settings.
\begin{flow}
  \item
    The Campaign Manager overrides respective default settings with values
    specified by the Operator.
\end{flow}

% Include all the relevant bib files.
% https://lsst-texmf.lsst.io/lsstdoc.html#bibliographies
%\bibliography{lsst,lsst-dm,refs_ads,refs,books}

\end{document}
# vim: ts=2 sw=2 et
